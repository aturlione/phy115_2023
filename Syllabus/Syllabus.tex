% Don't touch this %%%%%%%%%%%%%%%%%%%%%%%%%%%%%%%%%%%%%%%%%%%
\documentclass[12pt]{article}
\usepackage{fullpage}
\usepackage[left=1in,top=1in,right=1in,bottom=1in,headheight=3ex,headsep=3ex]{geometry}
\usepackage{graphicx}
\usepackage{float}
\usepackage{array}


\newcommand{\blankline}{\quad\pagebreak[2]}
%%%%%%%%%%%%%%%%%%%%%%%%%%%%%%%%%%%%%%%%%%%%%%%%%%%%%%%%%%%%%%

% Modify Course title, instructor name, semester here %%%%%%%%

\title{PHY115: Introduction to Applied Math and Physics}
\author{Spring 2023}
\date{}

%%%%%%%%%%%%%%%%%%%%%%%%%%%%%%%%%%%%%%%%%%%%%%%%%%%%%%%%%%%%%%

% Don't touch this %%%%%%%%%%%%%%%%%%%%%%%%%%%%%%%%%%%%%%%%%%%
\usepackage[sc]{mathpazo}
%\linespread{1.05} % Palatino needs more leading (space between lines)
\usepackage[T1]{fontenc}
\usepackage[mmddyyyy]{datetime}% http://ctan.org/pkg/datetime
\usepackage{advdate}% http://ctan.org/pkg/advdate
\newdateformat{syldate}{\twodigit{\THEMONTH}/\twodigit{\THEDAY}}
\newsavebox{\MONDAY}\savebox{\MONDAY}{Mon}% Mon
\newcommand{\week}[1]{%
%  \cleardate{mydate}% Clear date
% \newdate{mydate}{\the\day}{\the\month}{\the\year}% Store date
  \paragraph*{\kern-2ex\quad #1, \syldate{\today} - \AdvanceDate[4]\syldate{\today}:}% Set heading  \quad #1
%  \setbox1=\hbox{\shortdayofweekname{\getdateday{mydate}}{\getdatemonth{mydate}}{\getdateyear{mydate}}}%
  \ifdim\wd1=\wd\MONDAY
    \AdvanceDate[7]
  \else
    \AdvanceDate[7]
  \fi%
}
%\usepackage{setspace}
\usepackage{multicol}
%\usepackage{indentfirst}
\usepackage{fancyhdr,lastpage}
\usepackage{url}
\pagestyle{fancy}
\usepackage{hyperref}
\usepackage{lastpage}
\usepackage{amsmath}
\usepackage{layout}

\lhead{}
\chead{}
%%%%%%%%%%%%%%%%%%%%%%%%%%%%%%%%%%%%%%%%%%%%%%%%%%%%%%%%%%%%%%

% Modify header here %%%%%%%%%%%%%%%%%%%%%%%%%%%%%%%%%%%%%%%%%
\rhead{\footnotesize Text in header}

%%%%%%%%%%%%%%%%%%%%%%%%%%%%%%%%%%%%%%%%%%%%%%%%%%%%%%%%%%%%%%
% Don't touch this %%%%%%%%%%%%%%%%%%%%%%%%%%%%%%%%%%%%%%%%%%%
\lfoot{}
\cfoot{\small \thepage/\pageref*{LastPage}}
\rfoot{}

\usepackage{array, xcolor}
\usepackage{color,hyperref}
\definecolor{clemsonorange}{HTML}{EA6A20}
\hypersetup{colorlinks,breaklinks,linkcolor=clemsonorange,urlcolor=clemsonorange,anchorcolor=clemsonorange,citecolor=black}

\begin{document}

\maketitle

%\blankline

%\begin{tabular*}{.93\textwidth}{@{\extracolsep{\fill}}lr}

%%%%%%%%%%%%%%%%%%%%%%%%%%%%%%%%%%%%%%%%%%%%%%%%%%%%%%%%%%%%%%

% Modify information %%%%%%%%%%%%%%%%%%%%%%%%%%%%%%%%%%%%%%%%%
%E-mail: \texttt{anabela.turlione@digipen.edu}  \\

 %Office Hours: M 10-11:45am  &  Class Hours: T/Th 3-4:15pm \\

 %Office: ... & Class Room: ... \\
%% & \\
%Lab Room: ... & Lab Hours: W 3-5pm \\
%&  \\
%\hline
%\end{tabular*}

%\begin{figure*}
%\includegraphics[width=1.3\textwidth,angle=90]{Concept_map_315.pdf}
%\end{figure*}


\hrule



% First Section %%%%%%%%%%%%%%%%%%%%%%%%%%%%%%%%%%%%%%%%%%%%
\section*{General Information }

Class Schedule: Wednesdays 9:00am-12:00pm \\ 
\\
Class room: Auditorium\\
\\
Professor: Anabela Turlione\\
\\
Contact: anabela.turlione@digipen.edu - int:1029\\
\\
Class web page: PHY115 at distance.digipen.edu\\
\\
Office hours: by appointment\\


%\bigskip

%\noindent New paragraph. Bla bla bla ...

% Second Section %%%%%%%%%%%%%%%%%%%%%%%%%%%%%%%%%%%%%%%%%%%

\section*{Description}

This course examines the basic physics and mathematics governing natural phenomena such
as light, weight, inertia, friction, momentum, and thrust as a practical introduction to
applied math and physics. Students will explore geometry, trigonometry for cyclical motions,
and physical equations of motion for bodies moving under the influence of forces. With
these tools, students will develop a broader understanding of the impact of mathematics
and physics on their daily lives.


%\begin{itemize}
%\item Course notes available on Moodle. Books. Tech. Bla bla bla ...
%\end{itemize}

% Third Section %%%%%%%%%%%%%%%%%%%%%%%%%%%%%%%%%%%%%%%%%%%

\section*{Course Objectives and Learning Outcomes }
Upon a successfully completion of this course the students will gain a fundamental understanding of basic physical principles including:

\begin{enumerate}
\item Kinematics (linear motion, circular motion)
\item Mechanics (Newton's Laws, Rotation of Rigid bodies)
\item Light (phenomenological description)
\end{enumerate}


It will also provide skills in the geometry, trigonometry and algebra needed to
solve related problems. Applications to animation and every day life will be emphasized.

% Fourth Section %%%%%%%%%%%%%%%%%%%%%%%%%%%%%%%%%%%%%%%%%%%

\section*{Textbooks}

\begin{itemize}
\item 
	College Physics 7th edition, by Sears, Zemansky, Yound, Ed.
	Addison-Wesley
\end{itemize}

% Fifth Section %%%%%%%%%%%%%%%%%%%%%%%%%%%%%%%%%%%%%%%%%%%

\section*{Grading Policy}

The breakdown of the weighting of the Total Score will be as follows:

\begin{enumerate}
\item  	Assigments (20 $\%$)
\item   Midterm exam (40$\%$)
\item 	Final exam (40$\%$)

\end{enumerate}

The minimum grade to pass the subject is 70 \% 


\section*{Mechanisms and Procedures}

\begin{enumerate}
\item Before the class: To optimize the learning experience and the efficient use of our time, reading the relevant book sections before the material is taught
 is recommended. 
\item Attendance:  attendance in class is mandatory. If you are absent for 2 weeks or more you are considered to have withdrawn from the course. If you decide to
 drop the course it is your responsibility to follow the correct procedures.
\item No food is allowed in class. I strongly recommend to take notes during the class.
\item Working problems is essential in mastering the material. There will be approximately an assignment every two weeks.  At least, one of the assigments will be a 
programming assignments. Programming assignments must be submitted to Moodle before the deadline.
\item Late Policy: Late homework will not be accepted.
\item  Please fell free to send me an email whenever you need help.
\item A calculator is required for tests and homework. You can use any programming calculator without an Internet connection during any test or exam.
\item All exams are closed book. As this course is about understanding and not memorization, one sheet of notes is permitted for an exam with the formulas you 
consider. This sheet of notes must be handwritten by you, and no larger than a ‘normal’ (DIN4) piece of paper. Front and back of the page may be used.
\item Missing a test without a timely, valid excuse will result in a 0 score for the test. There are NO make up exams unless you have a compelling and well 
documented reason for missing a test. Notice that make ups are only considered under relevant medical, familiar or administrative situations that cannot be 
postponed.
\end{enumerate}

\section*{Rubrics and Assessment}

To get full credit you need to show all the important steps of your work which are:

\begin{enumerate}
\item Do a drawing of the schedule/body diagram in each problem.
\item Indicate the law/theory applied (or your reasoning).
\item Develop your calculus. The level of detail required is what your colleagues would need to see in order to understand your solution, without having to 
work it out for themselves.
\item Give the solution, specifying the unit of each magnitude and the direction of them (it's a vector). A penalty of 10% will be apply if the units are missing. 
\end{enumerate}

\begin{itemize}
\item Before submitting an assignment your grade is a 0, not a 100. This means that you obtain points for doing things right, and I do not subtract points from 
that non existing 100. 
\item Partial credit will be given only if your work is clearly presented and mostly correct.
\item If the process that you follow is correct but you arrive to a conclusion that is totally inconsistent with the theory learned in class, you will get a zero
 in that exercises with the note “misconception” attached. 
\item If an error is accumulative along an exercise, that will not penalize the rest of the exercise unless this means inconsistency with the theory learned in 
class. 
\item Any material covered in the course is valid for testing, including concepts covered in lecture, homework, or other communications and/or assigned work 
(as reading the textbook).
\item No messy tests/homework will be graded.
\end{itemize}

\section*{Relevance/Statement}

It is important to keep up with the material, to study regularly at home (at least 2 hours for every hour in class) and to do as many problems as you can 
(don’t limit yourself to the assigned or recommended problems, or merely the problems that are due).  

	You are welcome to work with other students, so long as the aim is furthering your understanding of the concepts and problem solving techniques. I am 
	happy to help work through problems, either in office hours or in class. Just remember, doing a problem yourself is very different from watching another 
	person do so. If you work together on problem sets, be sure to provide your own solution to every problem proving that you understand your writing. 
	In addition, some exam questions may be a resemblance to homework questions, so you’re encouraged to fully understand what you turn in. 
	Again, reading the relevant book sections before the material is taught is highly recommended. 


\subsection*{Last Day to Withdraw:    On Sunday 3/7/2022}

\vspace{5 mm}

\section*{Academic Integrity Policy}

Academic dishonesty in any form will not be tolerated in this course. Cheating, copying from any sources (including current or past students work, online sources 
or books), plagiarizing, or any other form of academic dishonesty (including doing someone else’s individual assignments) will result in, at the extreme minimum,
 a zero on the assignment in question, and could result in a failing grade in the course or even expulsion from DigiPen. Assisting others in cheating is prohibited 
 and will be equally punished.

\section*{Disability Support Services}

Students that have special needs due to medical issues, can apply for formal accommodations. 
The accommodations are student specific and are focused on helping the student to complete the 
learning process and achieve the goals in the course. Students that apply for accommodations for the first
 time should contact the Administration Office at 94 6365163 in order to start the process. Students that 
 have already contacted the Administration Office will be informed about the general considerations through
  their Academic Advisor. Additionally, students should talk to the teacher in order to be informed about the
   details of the accommodations in this particular course. 

%\subsection*{Class Structure}

%Bla bla bla ...

%\bigskip

%Bla bla bla ...

%\subsection*{Assessments}

%...

%\subsubsection*{Lecture}
%Bla bla ...

%\subsubsection*{Lab}
%...

%\subsubsection*{Final Exam and Class Project}

%Bla bla \textbf{Bla bla}.

%\subsection*{Grading Policy}
%The typical NC State grading scale will be used. I reserve the right to curve the scale dependent on overall class scores at the end of the semester. Any curve will only ever make it easier to obtain a certain letter grade. The grade will count the assessments using the following proportions:
%\begin{itemize}
%	\item \underline{\textbf{30\%}} of your grade will be determined by 2 in class midterm exams (15\% each).
%	\item \underline{\textbf{5\%}} of your grade will be determined by ...
%	\item \underline{\textbf{5\%}} ...
%    \item \underline{\textbf{10\%}}  ...
%	\item \underline{\textbf{15\%}} ...
%	\item \underline{\textbf{15\%}} ...
%\end{itemize}

% Add a figure %%%%%%%%%%%%%%%%%%%%%%%%%%%%%%%%%%%%%%%%%%%

%\begin{figure*}
%\includegraphics[width=1.3\textwidth,angle=90]{Concept_map_315.pdf}
%\end{figure*}

% Fifth Section %%%%%%%%%%%%%%%%%%%%%%%%%%%%%%%%%%%%%%%%%%%

%\newpage



% Course Schedule %%%%%%%%%%%%%%%%%%%%%%%%%%%%%%%%%%%%%%%%%%%

\section*{Outline and Tentative Dates}


\raggedright
\begin{center}

 \begin{tabular}{|c |l| l| l|} 


 \hline
Timeline & Topic & HW &\multicolumn{1}{p{3cm}|}{ Approximate book Section} \\ [0.5ex] 
 \hline\hline
 Week 1 &\multicolumn{1}{p{10cm}|}{ Operations with vectors, trigonometry, solving quadratic
equations.} &  &Ch1  \\ 
 \hline
  Week 2 & \multicolumn{1}{p{10cm}|}{Distance and velocity, Motion with constant velocity
Motion with constant acceleration}   &HW 1 &Ch 2  \\
 \hline
 Week 3 &Newton's Laws of motion & & Ch 4 \\
 \hline
 Week 4 & \multicolumn{1}{p{10cm}|}{Different kind of forces. Free bodies diagrams}  & HW 2 & Ch 4 \\
 \hline

 Week 5 & \multicolumn{1}{p{10cm}|}{ Motion in the plane, component of acceleration } & & Ch 3  \\
\hline

 Week 6 & \multicolumn{1}{p{10cm}|}{ Applying Newton's Laws } &  HW 3& Ch 3  \\
\hline

 Week 7 &\multicolumn{1}{p{10cm}|}{ \textbf{Midterm test} } 
 &  & Ch 5 \\ 

 \hline
 Week 8 &\multicolumn{1}{p{10cm}|}{ Impulse, Momentum and Collisions }&  HW 4 &Ch 8  \\ 
 \hline
 Week 9 & \multicolumn{1}{p{10cm}|}{CM and conservation of the linear momentum. }
 & & Ch 9 \\ 
 \hline
 Week 10 &\multicolumn{1}{p{10cm}|}{ Rotation of Rigid Bodies} & HW 5 &Ch 10  \\ 
 \hline
 Week 11 &\multicolumn{1}{p{10cm}|}{Dynamics of Rotational Motion }&  & Ch 34 \\ 
 \hline
 Week 12 &\multicolumn{1}{p{10cm}|}{Light: basic concepts}  & HW 6 & Ch 34 \\ 
 \hline
 Week 13 & \multicolumn{1}{p{10cm}|}{ Physics of shading} &  & Ch 36 \\ 
 \hline
 Week 14 &\multicolumn{1}{p{10cm}|}{Review before the Final test } &HW7   &  \\ 
 \hline

 Week 15 & \textbf{FINAL test}&  &  \\ 
 \hline
\end{tabular}

\end{center}


[This entire syllabus, particularly the time line, may be adjusted or changed at any time by the instructor.]

\end{document}


